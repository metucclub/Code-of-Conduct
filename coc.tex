\documentclass{article}
\usepackage[utf8]{inputenc}
\usepackage{indentfirst}
\title{Our Code of Conduct}
\author{METU Computer Club}

\begin{document}
\maketitle
\section*{Introduction}

Our code of conduct reflects our key values and aims to underline some rules we
deem necessary to maintain an inclusive safe space for our members of our
student club. In our student club, we try to avoid enforcing a strict set of
rules in an authoritarian manner; however, our past experiences have shown that
a bare minimum set of rules are required for the good of our club and its
members. Therefore, all members of Computer Club (cclub for short) are required
to read this document thoroughly and comply with our code of conduct.


If you have any questions or recommendations regarding our code of conduct,
please feel free to contact any board of directors or board of advisors member.
Keep in mind that this document might (and probably will) be subject to
additions to cover a wider array of topics.

\section*{Praise Coexistence!}

\subsubsection*{Who can join cclub?}
cclub welcomes any student who is fascinated by computer science and
engineering, or technology, in general. Students from any department can join
cclub and the members are not expected to have any knowledge of nor experience
in any related topic at all. Please note that cclub is a technical community of
students united by their love for computers, and you don't need to meet a
certain criterion to be a part of this community.

\subsubsection*{Accept people as they are. Be mindful of those around you.}
Always keep in mind that we do have members from different backgrounds.  Avoid
making statements that may offend other members or make them feel threatened
and do not discriminate against any member based on their ethnicity, religion,
age, sexual orientation, gender identity and expression, body size, or the
group, class, or category to which they are perceived to belong.


We value freedom of speech only to the extent it does not cause any harm to
anyone physically nor psychologically. Dark humor or other kinds of potentially
offensive statements are to be made only if you are \textbf{absolutely sure}
that no member will feel discriminated against. It is always a good practice to
get to know everyone before making \textit{risky} jokes, i.e.\ refrain from
offending members proactively.  A reactive approach depending on the feedbacks
from others is not that good of an idea since you would have already offended
some members, and some offended members might just don't want to out themselves
in such cases.

\section*{Respect The Rules of the Club Room.}
cclub has a club room in the department of computer engineering building. Our
members can use that room to study, socialize, or have a quick rest between two
classes. To keep our club room nice and tidy, and let our members fully benefit
from our club room, we have some rules. Essentially, there are three rules from
which the others are derived:
\begin{enumerate}
		\item Do not disrupt or disturb other members in the room.
		\item Do not disrupt or disturb other people outside the room.
		\item Keep the room nice and clean.
\end{enumerate}

There is a more comprehensive yet still incomplete list of club room rules available.

\section*{Maintain Academic Integrity}
cclub condemns academic dishonesty and should never be a mean to a dishonest end. Therefore, we strictly forbid using cclub as a medium for cheating. If you are having problems with your homeworks and looking for someone to do your homework, this is not the place to seek for that kind of help.


\end{document}
